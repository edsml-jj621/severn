%% Generated by Sphinx.
\def\sphinxdocclass{report}
\documentclass[letterpaper,10pt,english]{sphinxmanual}
\ifdefined\pdfpxdimen
   \let\sphinxpxdimen\pdfpxdimen\else\newdimen\sphinxpxdimen
\fi \sphinxpxdimen=.75bp\relax
\ifdefined\pdfimageresolution
    \pdfimageresolution= \numexpr \dimexpr1in\relax/\sphinxpxdimen\relax
\fi
%% let collapsible pdf bookmarks panel have high depth per default
\PassOptionsToPackage{bookmarksdepth=5}{hyperref}

\PassOptionsToPackage{warn}{textcomp}
\usepackage[utf8]{inputenc}
\ifdefined\DeclareUnicodeCharacter
% support both utf8 and utf8x syntaxes
  \ifdefined\DeclareUnicodeCharacterAsOptional
    \def\sphinxDUC#1{\DeclareUnicodeCharacter{"#1}}
  \else
    \let\sphinxDUC\DeclareUnicodeCharacter
  \fi
  \sphinxDUC{00A0}{\nobreakspace}
  \sphinxDUC{2500}{\sphinxunichar{2500}}
  \sphinxDUC{2502}{\sphinxunichar{2502}}
  \sphinxDUC{2514}{\sphinxunichar{2514}}
  \sphinxDUC{251C}{\sphinxunichar{251C}}
  \sphinxDUC{2572}{\textbackslash}
\fi
\usepackage{cmap}
\usepackage[T1]{fontenc}
\usepackage{amsmath,amssymb,amstext}
\usepackage{babel}



\usepackage{tgtermes}
\usepackage{tgheros}
\renewcommand{\ttdefault}{txtt}



\usepackage[Bjarne]{fncychap}
\usepackage{sphinx}

\fvset{fontsize=auto}
\usepackage{geometry}


% Include hyperref last.
\usepackage{hyperref}
% Fix anchor placement for figures with captions.
\usepackage{hypcap}% it must be loaded after hyperref.
% Set up styles of URL: it should be placed after hyperref.
\urlstyle{same}


\usepackage{sphinxmessages}




\title{Flood Tool}
\date{Jun 03, 2022}
\release{}
\author{Team Severn}
\newcommand{\sphinxlogo}{\vbox{}}
\renewcommand{\releasename}{}
\makeindex
\begin{document}

\ifdefined\shorthandoff
  \ifnum\catcode`\=\string=\active\shorthandoff{=}\fi
  \ifnum\catcode`\"=\active\shorthandoff{"}\fi
\fi

\pagestyle{empty}
\sphinxmaketitle
\pagestyle{plain}
\sphinxtableofcontents
\pagestyle{normal}
\phantomsection\label{\detokenize{index::doc}}


\sphinxAtStartPar
This package implements a flood risk prediction and visualization tool.

\sphinxAtStartPar
This document contains the latest information about Flood Tool package, which is automatically generated on the remote Linux virtual machine hosted by GitHub.


\chapter{Installation Instructions}
\label{\detokenize{index:installation-instructions}}
\sphinxAtStartPar
To install the module \sphinxcode{\sphinxupquote{flood\_tools}} clone the respository to local by running:

\sphinxAtStartPar
\sphinxcode{\sphinxupquote{git clone https://github.com/ese\sphinxhyphen{}msc\sphinxhyphen{}2021/ads\sphinxhyphen{}deluge\sphinxhyphen{}severn.git}}.

\sphinxAtStartPar
Then navigateto the root of the repository that you cloned and install all the packages needed by
running:

\sphinxAtStartPar
\sphinxcode{\sphinxupquote{pip install \sphinxhyphen{}r requirements.txt}}

\sphinxAtStartPar
\sphinxcode{\sphinxupquote{pip install \sphinxhyphen{}e .}}.


\chapter{Usage guide}
\label{\detokenize{index:usage-guide}}
\sphinxAtStartPar
\sphinxcode{\sphinxupquote{python flood\_tool/main.py {[}\sphinxhyphen{}h{]} \sphinxhyphen{}t LABEL\_TYPE \sphinxhyphen{}f UNLABELLED\_FILE {[}\sphinxhyphen{}m METHOD{]} {[}\sphinxhyphen{}s SAVE{]}}}


\begin{savenotes}\sphinxattablestart
\centering
\begin{tabulary}{\linewidth}[t]{|T|T|}
\hline
\sphinxstyletheadfamily 
\sphinxAtStartPar
Options
&\sphinxstyletheadfamily 
\sphinxAtStartPar
Description
\\
\hline
\sphinxAtStartPar
\sphinxhyphen{}h, \textendash{}help
&
\sphinxAtStartPar
help message
\\
\hline
\sphinxAtStartPar
\sphinxhyphen{}t, \textendash{}label\_type
&
\sphinxAtStartPar
Type of labelling.
| \sphinxhyphen{}t flood\_risk
| \sphinxhyphen{}t house\_price
\\
\hline
\sphinxAtStartPar
\sphinxhyphen{}f postcodes.csv
&
\sphinxAtStartPar
Unlabelled postcodes file.
| \sphinxhyphen{}f postcodes.csv
\\
\hline
\sphinxAtStartPar
\sphinxhyphen{}m METHOD, \textendash{}method
&
\sphinxAtStartPar
(optional)

\sphinxAtStartPar
Flood Risk
| \sphinxstyleemphasis{default knn}
| \sphinxhyphen{}m dt:  Decision Tree
| \sphinxhyphen{}m knn:  KNN
| \sphinxhyphen{}m rmdf:  Random Forest
| \sphinxhyphen{}m ada:  AdaBoost

\sphinxAtStartPar
House Price
| \sphinxstyleemphasis{default rfr}
| \sphinxhyphen{}m lr:  Linear Regression
| \sphinxhyphen{}m dt:  Dscision Tree
| \sphinxhyphen{}m rfr: Random Forest Regression
| \sphinxhyphen{}m sv:  SVR Support Vector Regression
\\
\hline
\sphinxAtStartPar
\sphinxhyphen{}s, \textendash{}save
&
\sphinxAtStartPar
(optional)
\sphinxstyleemphasis{default labelled.csv}
output filename / path
\\
\hline
\end{tabulary}
\par
\sphinxattableend\end{savenotes}


\chapter{Geodetic Transformations}
\label{\detokenize{index:geodetic-transformations}}
\sphinxAtStartPar
For historical reasons, multiple coordinate systems exist in in current use in
British mapping circles. The Ordnance Survey has been mapping the British Isles
since the 18th Century and the last major retriangulation from 1936\sphinxhyphen{}1962 produced
the Ordance Survey National Grid (otherwise known as \sphinxstylestrong{OSGB36}), which defined
latitude and longitude for all points across the island of Great Britain %
\begin{footnote}[1]\sphinxAtStartFootnote
A guide to coordinate systems in Great Britain, Ordnance Survey
%
\end{footnote}.
For convenience, a standard Transverse Mercator projection %
\begin{footnote}[2]\sphinxAtStartFootnote
Map projections \sphinxhyphen{} A Working Manual, John P. Snyder, \sphinxurl{https://doi.org/10.3133/pp1395}
%
\end{footnote} was also defined,
producing a notionally flat 2D gridded surface, with gradations called eastings
and northings. The scale for these gradations was identified with metres, which
allowed local distances to be defined with a fair degree of accuracy.

\sphinxAtStartPar
The OSGB36 datum is based on the Airy Ellipsoid of 1830, which defines
semimajor axes for its model of the earth, \(a\) and \(b\), a scaling
factor \(F_0\) and ellipsoid height, \(H\).
\begin{equation*}
\begin{split}a &= 6377563.396, \\
b &= 6356256.910, \\
F_0 &= 0.9996012717, \\
H &= 24.7.\end{split}
\end{equation*}
\sphinxAtStartPar
The point of origin for the transverse Mercator projection is defined in the
Ordnance Survey longitude\sphinxhyphen{}latitude and easting\sphinxhyphen{}northing coordinates as
\begin{equation*}
\begin{split}\phi^{OS}_0 &= 49^\circ \mbox{ north}, \\
\lambda^{OS}_0 &= 2^\circ \mbox{ west}, \\
E^{OS}_0 &= 400000 m, \\
N^{OS}_0 &= -100000 m.\end{split}
\end{equation*}
\sphinxAtStartPar
More recently, the world has gravitated towards the use of satellite based GPS
equipment, which uses the (globally more appropriate) World Geodetic System
1984 (also known as \sphinxstylestrong{WGS84}). This datum uses a different ellipsoid, which offers a
better fit for a global coordinate system (as well as North America). Its key
properties are:
\begin{equation*}
\begin{split}a_{WGS} &= 6378137,, \\
b_{WGS} &= 6356752.314, \\
F_0 &= 0.9996.\end{split}
\end{equation*}
\sphinxAtStartPar
For a given point on the WGS84 ellipsoid, an approximate mapping to the
OSGB36 datum can be found using a Helmert transformation %
\begin{footnote}[3]\sphinxAtStartFootnote
Computing Helmert transformations, G Watson, \sphinxurl{http://www.maths.dundee.ac.uk/gawatson/helmertrev.pdf}
%
\end{footnote},
\begin{equation*}
\begin{split}\mathbf{x}^{OS} = \mathbf{t}+\mathbf{M}\mathbf{x}^{WGS}.\end{split}
\end{equation*}
\sphinxAtStartPar
Here \(\mathbf{x}\) denotes a coordinate in Cartesian space (i.e in 3D)
as given by the (invertible) transformation
\begin{equation*}
\begin{split}\nu &= \frac{aF_0}{\sqrt{1-e^2\sin^2(\phi^{OS})}} \\
x &= (\nu+H) \sin(\lambda)\cos(\phi) \\
y &= (\nu+H) \cos(\lambda)\cos(\phi) \\
z &= ((1-e^2)\nu+H)\sin(\phi)\end{split}
\end{equation*}
\sphinxAtStartPar
and the transformation parameters are
\begin{eqnarray*}
\mathbf{t} &= \left(\begin{array}{c}
-446.448\\ 125.157\\ -542.060
\end{array}\right),\\
\mathbf{M} &= \left[\begin{array}{ c c c }
1+s& -r_3& r_2\\
r_3 & 1+s & -r_1 \\
-r_2 & r_1 & 1+s
\end{array}\right], \\
s &= 20.4894\times 10^{-6}, \\
\mathbf{r} &= [0.1502'', 0.2470'', 0.8421''].
\end{eqnarray*}
\sphinxAtStartPar
Given a latitude, \(\phi^{OS}\) and longitude, \(\lambda^{OS}\) in the
OSGB36 datum, easting and northing coordinates, \(E^{OS}\) \& \(N^{OS}\)
can then be calculated using the following formulae (see “A guide to coordinate
systems in Great Britain, Appendix C1):
\begin{equation*}
\begin{split}\rho &= \frac{aF_0(1-e^2)}{\left(1-e^2\sin^2(\phi^{OS})\right)^{\frac{3}{2}}} \\
\eta &= \sqrt{\frac{\nu}{\rho}-1} \\
M &= bF_0\left[\left(1+n+\frac{5}{4}n^2+\frac{5}{4}n^3\right)(\phi^{OS}-\phi^{OS}_0)\right. \\
&\quad-\left(3n+3n^2+\frac{21}{8}n^3\right)\sin(\phi-\phi_0)\cos(\phi^{OS}+\phi^{OS}_0) \\
&\quad+\left(\frac{15}{8}n^2+\frac{15}{8}n^3\right)\sin(2(\phi^{OS}-\phi^{OS}_0))\cos(2(\phi^{OS}+\phi^{OS}_0)) \\
&\left.\quad-\frac{35}{24}n^3\sin(3(\phi-\phi_0))\cos(3(\phi^{OS}+\phi^{OS}_0))\right] \\
I &= M + N^{OS}_0 \\
II &= \frac{\nu}{2}\sin(\phi^{OS})\cos(\phi^{OS}) \\
III &= \frac{\nu}{24}\sin(\phi^{OS})cos^3(\phi^{OS})(5-\tan^2(phi^{OS})+9\eta^2) \\
IIIA &= \frac{\nu}{720}\sin(\phi^{OS})cos^5(\phi^{OS})(61-58\tan^2(\phi^{OS})+\tan^4(\phi^{OS})) \\
IV &= \nu\cos(\phi^{OS}) \\
V &= \frac{\nu}{6}\cos^3(\phi^{OS})\left(\frac{\nu}{\rho}-\tan^2(\phi^{OS})\right) \\
VI &= \frac{\nu}{120}\cos^5(\phi^{OS})(5-18\tan^2(\phi^{OS})+\tan^4(\phi^{OS}) \\
&\quad+14\eta^2-58\tan^2(\phi^{OS})\eta^2) \\
E^{OS} &= E^{OS}_0+IV(\lambda^{OS}-\lambda^{OS}_0)+V(\lambda-\lambda^{OS}_0)^3+VI(\lambda^{OS}-\lambda^{OS}_0)^5 \\
N^{OS} &= I + II(\lambda^{OS}-\lambda^{OS}_0)^2+III(\lambda-\lambda^{OS}_0)^4+IIIA(\lambda^{OS}-\lambda^{OS}_0)^6\end{split}
\end{equation*}
\sphinxAtStartPar
The inverse transformation can be generated iteratively using a fixed point process:
\begin{enumerate}
\sphinxsetlistlabels{\arabic}{enumi}{enumii}{}{.}%
\item {} 
\sphinxAtStartPar
Set \(M=0\) and \(\phi^{OS} = \phi_0^{OS}\).

\item {} 
\sphinxAtStartPar
Update \(\phi_{i+1}^{OS} = \frac{N-N_0-M}{aF_0}+\phi_i^{OS}\)

\item {} 
\sphinxAtStartPar
Calculate \(M\) using the formula above.

\item {} 
\sphinxAtStartPar
If \(\textrm{abs}(N-N_0-M)> 0.01 mm\) go to 2, otherwise halt.

\end{enumerate}

\sphinxAtStartPar
With \(M\) calculated we now improve our estimate of \(\phi^{OS}\). First calculate
\(\nu\), \(\rho\) and \(\eta\) using our previous formulae. Next
\begin{equation*}
\begin{split}VII &= \frac{\tan(\phi^{OS})}{2\rho\nu},\\
VIII &= \frac{\tan(\phi^{OS})}{24\rho\nu^3}\left(5+3\tan^2(\phi^{OS})+\eta^2-9\tan^2(\phi^{OS})\eta^2\right),\\
IX &= \frac{\tan(\phi^{OS})}{720\rho\nu^5}\left(61+90\tan^2(\phi^{OS})+45\tan^4(\phi^{OS})\right),\\
X &= \frac{\sec\phi^{OS}}{\nu}, \\
XI &= \frac{\sec\phi^{OS}}{6\nu^3}\left(\frac{\nu}{\rho}+2\tan^2(\phi^{OS})\right), \\
XII &= \frac{\sec\phi^{OS}}{120\nu^5}\left(5+28\tan^2(\phi^{OS})+24\tan^4(\phi^{OS})\right), \\
XIIA &= \frac{\sec\phi^{OS}}{5040\nu^5}\left(61+662\tan^2(\phi^{OS})+1320\tan^4(\phi^{OS})+720\tan^6(\phi^{OS})\right).\end{split}
\end{equation*}
\sphinxAtStartPar
Finally, the corrected values for \(\phi^{OS}\) and \(\lambda^{OS}\) are:
\begin{equation*}
\begin{split}\phi_{\textrm{final}}^{OS} &= \phi^{OS} -VII(E-E_0)^2 +VIII(E-E_0)^4 -IX(E-E_0)^6, \\
\lambda_{\textrm{final}}^{OS} &= \lambda_0^{OS}+X(E-E_0)-XI(E-E_0)^3+ XII(E-E_0)^5-XII(E-E_0)^7.\end{split}
\end{equation*}

\chapter{Classifier choice}
\label{\detokenize{index:classifier-choice}}
\sphinxAtStartPar
In order to give more options for who would use this module to make predictions about the flood probability, we provided several trained classifiers:
\begin{itemize}
\item {} 
\sphinxAtStartPar
Decision Tree

\item {} 
\sphinxAtStartPar
K\sphinxhyphen{}Nearest Neighbors

\item {} 
\sphinxAtStartPar
Random Forest

\item {} 
\sphinxAtStartPar
AdaBoost

\end{itemize}
\begin{quote}\begin{description}
\sphinxlineitem{Decision Tree}
\sphinxAtStartPar
The decision tree classifier (Pang\sphinxhyphen{}Ning et al., 2006) creates the classification model by building a decision tree.Each node in the tree specifies a test on an attribute, each branch descending from that node corresponds to one of the possible values for that attribute.

\sphinxlineitem{K\sphinxhyphen{}Nearest Neighbors}
\sphinxAtStartPar
In statistics, the k\sphinxhyphen{}nearest neighbors algorithm (k\sphinxhyphen{}NN) is a non\sphinxhyphen{}parametric classification method first developed by Evelyn Fix and Joseph Hodges in 1951, and later expanded by Thomas Cover. It is used for classification and regression. In both cases, the input consists of the k closest training examples in a data set.

\sphinxlineitem{Random Forest}
\sphinxAtStartPar
As the name suggests, “Random Forest is a classifier that contains a number of decision trees on various subsets of the given dataset and takes the average to improve the predictive accuracy of that dataset.” Instead of relying on one decision tree, the random forest takes the prediction from each tree and based on the majority votes of predictions, and it predicts the final output.

\sphinxlineitem{AdaBoost}
\sphinxAtStartPar
An AdaBoost classifier is a meta\sphinxhyphen{}estimator that begins by fitting a classifier on the original dataset and then fits additional copies of the classifier on the same dataset but where the weights of incorrectly classified instances are adjusted such that subsequent classifiers focus more on difficult cases.

\end{description}\end{quote}


\chapter{Regressor choice}
\label{\detokenize{index:regressor-choice}}\begin{quote}\begin{description}
\sphinxlineitem{Just like the classifier choice, we provided many trained regressors for users as well}
\end{description}\end{quote}
\begin{itemize}
\item {} 
\sphinxAtStartPar
Linear Regression

\item {} 
\sphinxAtStartPar
Decision Tree Regressor

\item {} 
\sphinxAtStartPar
Random Forest Regressor

\item {} 
\sphinxAtStartPar
Supprot Vector Regressor

\end{itemize}
\begin{quote}\begin{description}
\sphinxlineitem{Linear Regression}
\sphinxAtStartPar
In statistics, linear regression is a linear approach for modelling the relationship between a scalar response and one or more explanatory variables. The case of one explanatory variable is called simple linear regression; for more than one, the process is called multiple linear regression.

\sphinxlineitem{Decision Tree Regressor}
\sphinxAtStartPar
Decision tree builds regression or classification models in the form of a tree structure. It breaks down a dataset into smaller and smaller subsets while at the same time an associated decision tree is incrementally developed. The final result is a tree with decision nodes and leaf nodes.

\sphinxlineitem{Random Forest Regressor}
\sphinxAtStartPar
Random Forest Regression is a supervised learning algorithm that uses ensemble learning method for regression. Ensemble learning method is a technique that combines predictions from multiple machine learning algorithms to make a more accurate prediction than a single model.

\sphinxlineitem{Supprot Vector Regressor}
\sphinxAtStartPar
Support Vector Regression is a supervised learning algorithm that is used to predict discrete values. Support Vector Regression uses the same principle as the SVMs. The basic idea behind SVR is to find the best fit line. In SVR, the best fit line is the hyperplane that has the maximum number of points.

\end{description}\end{quote}


\chapter{Data visualization}
\label{\detokenize{index:data-visualization}}
\sphinxAtStartPar
Fig.1 shows all the information we have about the given postcode using a popup message (a html table). It shows the total rainfall in mm and maximum river level in mASD of that location on a wet day (you can specify a typical day if you want in the code). Both of rainfall and river level data are retrieved from the closest monitoring station from that postcode. And it shows the maximum rainfall class of the day. The rainfall classifier is based on the table shown below. The following information, flood event probability, property value, and flood risk are all predicted by our model.
\begin{quote}

\begin{figure}[htbp]
\centering
\capstart

\noindent\sphinxincludegraphics[width=500\sphinxpxdimen]{{pic1}.png}
\caption{Indivitual postcode location with corresponding Flood and rainfall information}\label{\detokenize{index:id7}}\end{figure}
\end{quote}

\sphinxAtStartPar
Fig.2 visualizes the sreading of predicted flood probability among the selected postcodes from the test file. The flood event probabilities for the areas were generated from the riskLabel classifier defined in tool.py. The color map was chosen as perceptional sequential color range from orange to dark red(which representes low to high value) with a sequential gradient. The individual probability of flood for each postcode could be displayed after a mouse click. The map shows that north\sphinxhyphen{}eastern coastal area of the UK has a generally higher flood event probability than western area. There are a few darker points showing the riskest areas of flood including Windsor, Silsden, Gainsborough and etc. They have been observed with a common feature that they are located next to a river or they are in costal areas.
\begin{quote}

\begin{figure}[htbp]
\centering
\capstart

\noindent\sphinxincludegraphics[width=500\sphinxpxdimen]{{pic2}.png}
\caption{The spreading of predicted flood probability among the selected postcode}\label{\detokenize{index:id8}}\end{figure}
\end{quote}

\sphinxAtStartPar
Fig.3 and Fig.4 visualizes the heat map for maximum daily rainfall and riverlevels at a specific date. The first map shows the rainfall heatmap and the second map shows the riverlevel heatmap. The color map was chosen as perceptional sequential color from lime to blue(low to high) in the rainfall heatmap. The rainfall heatmap shows that the rainfall in the northern part around Leads and Manchester is higher than other parts indicated by the darker color. In the riverlevel heatmap, the map shows that the riverlevel is higher in Southern Uk near. These patterns match the flood probability plot that higher flood risk occurs in Northeast and Southeast.
\begin{quote}

\begin{figure}[htbp]
\centering
\capstart

\noindent\sphinxincludegraphics[width=500\sphinxpxdimen]{{pic3}.png}
\caption{The heatmap for the total daily rainfall and riverlevel for the selected postcodes(1)}\label{\detokenize{index:id9}}\end{figure}

\begin{figure}[htbp]
\centering
\capstart

\noindent\sphinxincludegraphics[width=500\sphinxpxdimen]{{pic4}.png}
\caption{The heatmap for the total daily rainfall and riverlevel for the selected postcodes(2)}\label{\detokenize{index:id10}}\end{figure}
\end{quote}

\sphinxAtStartPar
Fig.5 visualizes the rainfall heatmap variation in a 24 hour period on 5th May 2021. As we go through the animation, we can see that there is little rainfall variation in the South\sphinxhyphen{}eastern part of the UK among the 24 hours timeline, while the rainfall patterns in the North\sphinxhyphen{}eastern have a large rainfall variation in the same period. This means the reliability of rainfall prediction in the North is lower than other areas, as its’rapid rainfall variations could result in uncerntainties in its rainfall pattern estimation. This could be explained by the relatively extreme weather in high latitude area of the UK. We can also see from the animation that the rainfall in the mid day is lower than in the morning and evening.
\begin{quote}

\begin{figure}[htbp]
\centering
\capstart

\noindent\sphinxincludegraphics[width=500\sphinxpxdimen]{{pic5}.png}
\caption{The heatmap variation animation with a timeline for 24 hours}\label{\detokenize{index:id11}}\end{figure}
\end{quote}

\sphinxAtStartPar
Fig.6 shows the predicted property value for the selected postcode (in postcodes\_unlabelled.csv) around the UK. There are four categories, including 0\sphinxhyphen{}250k, 250k\sphinxhyphen{}375k, 375k\sphinxhyphen{}500k, and larger than 500k, corresponding to four different colors (light blue, cadet blue, blue and dark blue) respectively. It is found that the distribution and trend are very clear. From south to north, the property value is decreasing gradually in general. Most of the properties near the Great London area larger than 500k (dark blue). The other major cities like Birmingham and Nottingham in the middle of the UK have the property value around 250\sphinxhyphen{}375k (cadet blue). As for the north of the UK like Newcastle, Leeds and Manchester, the property value in this area is below 250k (light blue).
\begin{quote}

\begin{figure}[htbp]
\centering
\capstart

\noindent\sphinxincludegraphics[width=500\sphinxpxdimen]{{pic6}.png}
\caption{The spread of predicted house price for the selected postcodes}\label{\detokenize{index:id12}}\end{figure}
\end{quote}

\sphinxAtStartPar
Fig.7 shows the property value in the sample locations given in postcodes\_sampled.csv around the UK. The mapping strategy is the same as the map showing the predicted property value. Four different colors (light blue, cadet blue, blue and dark blue) shows four categories. The general distribution and trend are similar to our predicted one. However, as the locations is much more than the predicted one, there are more data points providing the property values. It is found some property values are below 500k in the south of the UK, while some are above 500k in the north of the UK, although only account for a small number of it.
\begin{quote}

\begin{figure}[htbp]
\centering
\capstart

\noindent\sphinxincludegraphics[width=500\sphinxpxdimen]{{pic7}.png}
\caption{The spreading of median house price in sampled csv}\label{\detokenize{index:id13}}\end{figure}
\end{quote}

\sphinxAtStartPar
Fig.8 shows the actual flood event probability in the sample locations given in postcodes\_sampled.csv around the UK. The mapping strategy is the same as the map showing the predicted property value.This map shows that high flood probability occurs near the north\sphinxhyphen{}eartern coastal area of the UK, which shows a similar pattern as the predicted flood event probability map using unlabelled data.
\begin{quote}

\begin{figure}[htbp]
\centering
\capstart

\noindent\sphinxincludegraphics[width=500\sphinxpxdimen]{{pic8}.png}
\caption{The spreading of flood probability given in sampled csv}\label{\detokenize{index:id14}}\end{figure}
\end{quote}


\chapter{Data analysis}
\label{\detokenize{index:data-analysis}}
\sphinxAtStartPar
For the flood class prediction, the x axis of the histogram means 10 different flood classes from 1 to 10. 1 indicates it has only 0.01\% probability that the place encounters flood, while class 10 show 5\% probability. So the lowest class 1 expects one event in 1000 years (or longer) and the highest risk class 10 expects one event in 20 years (or sooner). Most places are predicted a flood could hardly appear. Several postcode areas have class 6\sphinxhyphen{}9, and we should pay more attention to these areas’ river and rainfall data.
\begin{quote}

\begin{figure}[htbp]
\centering
\capstart

\noindent\sphinxincludegraphics[width=500\sphinxpxdimen]{{pic9}.png}
\caption{The predicted flood class distribution}\label{\detokenize{index:id15}}\end{figure}
\end{quote}

\sphinxAtStartPar
The x axis shows the predicted median house values, and the y axis indicate the number of properties. Using decision tree and random forest methods, the distribution of predicted median price is more divergent, the highest price reach 7e6 and 5e6 recpectively, while through linear regression and sv regressors, the predicted value is in the range of 100000 to 700000 pounds.
\begin{quote}

\begin{figure}[htbp]
\centering
\capstart

\noindent\sphinxincludegraphics[width=500\sphinxpxdimen]{{pic10}.png}
\caption{The predicted property value distribution}\label{\detokenize{index:id16}}\end{figure}
\end{quote}

\sphinxAtStartPar
Combining the above three figures, we find the distribution of annual flood risk has the similar pattern with the flood class distribution. Hence, we conclude that the flood risk is mainly decided by the flood class.
\begin{quote}

\begin{figure}[htbp]
\centering
\capstart

\noindent\sphinxincludegraphics[width=500\sphinxpxdimen]{{pic11}.png}
\caption{The predicted annual flood risk distribution}\label{\detokenize{index:id17}}\end{figure}
\end{quote}


\chapter{Function APIs}
\label{\detokenize{index:module-flood_tool}}\label{\detokenize{index:function-apis}}\index{module@\spxentry{module}!flood\_tool@\spxentry{flood\_tool}}\index{flood\_tool@\spxentry{flood\_tool}!module@\spxentry{module}}
\sphinxAtStartPar
Python flood risk analysis tool
\index{Tool (class in flood\_tool)@\spxentry{Tool}\spxextra{class in flood\_tool}}

\begin{fulllineitems}
\phantomsection\label{\detokenize{index:flood_tool.Tool}}
\pysigstartsignatures
\pysiglinewithargsret{\sphinxbfcode{\sphinxupquote{class\DUrole{w}{  }}}\sphinxcode{\sphinxupquote{flood\_tool.}}\sphinxbfcode{\sphinxupquote{Tool}}}{\emph{\DUrole{n}{postcode\_file}\DUrole{o}{=}\DUrole{default_value}{\textquotesingle{}\textquotesingle{}}}, \emph{\DUrole{n}{sample\_labels}\DUrole{o}{=}\DUrole{default_value}{\textquotesingle{}\textquotesingle{}}}, \emph{\DUrole{n}{household\_file}\DUrole{o}{=}\DUrole{default_value}{\textquotesingle{}\textquotesingle{}}}}{}
\pysigstopsignatures
\sphinxAtStartPar
Class to interact with a postcode database file.
\begin{quote}\begin{description}
\sphinxlineitem{Parameters}\begin{itemize}
\item {} 
\sphinxAtStartPar
\sphinxstyleliteralstrong{\sphinxupquote{postcode\_file}} (\sphinxstyleliteralemphasis{\sphinxupquote{str}}\sphinxstyleliteralemphasis{\sphinxupquote{, }}\sphinxstyleliteralemphasis{\sphinxupquote{optional}}) \textendash{} Filename of a .csv file containing geographic location
data for postcodes.

\item {} 
\sphinxAtStartPar
\sphinxstyleliteralstrong{\sphinxupquote{sample\_labels}} (\sphinxstyleliteralemphasis{\sphinxupquote{str}}\sphinxstyleliteralemphasis{\sphinxupquote{, }}\sphinxstyleliteralemphasis{\sphinxupquote{optional}}) \textendash{} Filename of a .csv file containing sample data on property
values and flood risk labels.

\item {} 
\sphinxAtStartPar
\sphinxstyleliteralstrong{\sphinxupquote{household\_file}} (\sphinxstyleliteralemphasis{\sphinxupquote{str}}\sphinxstyleliteralemphasis{\sphinxupquote{, }}\sphinxstyleliteralemphasis{\sphinxupquote{optional}}) \textendash{} Filename of a .csv file containing information on households
by postcode.

\end{itemize}

\end{description}\end{quote}
\index{get\_annual\_flood\_risk() (flood\_tool.Tool method)@\spxentry{get\_annual\_flood\_risk()}\spxextra{flood\_tool.Tool method}}

\begin{fulllineitems}
\phantomsection\label{\detokenize{index:flood_tool.Tool.get_annual_flood_risk}}
\pysigstartsignatures
\pysiglinewithargsret{\sphinxbfcode{\sphinxupquote{get\_annual\_flood\_risk}}}{\emph{\DUrole{n}{postcodes}}, \emph{\DUrole{n}{risk\_labels}\DUrole{o}{=}\DUrole{default_value}{None}}}{}
\pysigstopsignatures
\sphinxAtStartPar
Risk is defined here as a damage coefficient multiplied by the
value under threat multiplied by the probability of an event.
\begin{quote}\begin{description}
\sphinxlineitem{Parameters}\begin{itemize}
\item {} 
\sphinxAtStartPar
\sphinxstyleliteralstrong{\sphinxupquote{postcodes}} (\sphinxstyleliteralemphasis{\sphinxupquote{sequence of strs}}) \textendash{} Sequence of postcodes.

\item {} 
\sphinxAtStartPar
\sphinxstyleliteralstrong{\sphinxupquote{risk\_labels}} (\sphinxstyleliteralemphasis{\sphinxupquote{pandas.Series}}\sphinxstyleliteralemphasis{\sphinxupquote{ (}}\sphinxstyleliteralemphasis{\sphinxupquote{optional}}\sphinxstyleliteralemphasis{\sphinxupquote{)}}) \textendash{} Series containing flood probability classifiers, as
predicted by get\_flood\_probability.

\end{itemize}

\sphinxlineitem{Returns}
\sphinxAtStartPar
Series of total annual flood risk estimates indexed by locations.

\sphinxlineitem{Return type}
\sphinxAtStartPar
pandas.Series

\end{description}\end{quote}

\end{fulllineitems}

\index{get\_combined\_data() (flood\_tool.Tool method)@\spxentry{get\_combined\_data()}\spxextra{flood\_tool.Tool method}}

\begin{fulllineitems}
\phantomsection\label{\detokenize{index:flood_tool.Tool.get_combined_data}}
\pysigstartsignatures
\pysiglinewithargsret{\sphinxbfcode{\sphinxupquote{get\_combined\_data}}}{}{}
\pysigstopsignatures
\sphinxAtStartPar
Get lat, log, flood\_prob and annual\_flood\_risk from unlabelled
postcode file
\begin{quote}\begin{description}
\sphinxlineitem{Return type}
\sphinxAtStartPar
pandas.DataFrame

\end{description}\end{quote}
\subsubsection*{Example}

\begin{sphinxVerbatim}[commandchars=\\\{\}]
\PYG{g+gp}{\PYGZgt{}\PYGZgt{}\PYGZgt{} }\PYG{n}{get\PYGZus{}combined\PYGZus{}data}\PYG{p}{(}\PYG{p}{)}
\PYG{g+go}{         latitude  longitude  riskLabel  annualFloodRisk  floodProb}
\PYG{g+go}{YO62 4LS  54.146810  \PYGZhy{}0.966109          1         1.180043     0.00001}
\PYG{g+go}{DE2 3DA   52.872902  \PYGZhy{}1.496148          1         1.586082     0.00001}
\end{sphinxVerbatim}

\end{fulllineitems}

\index{get\_easting\_northing() (flood\_tool.Tool method)@\spxentry{get\_easting\_northing()}\spxextra{flood\_tool.Tool method}}

\begin{fulllineitems}
\phantomsection\label{\detokenize{index:flood_tool.Tool.get_easting_northing}}
\pysigstartsignatures
\pysiglinewithargsret{\sphinxbfcode{\sphinxupquote{get\_easting\_northing}}}{\emph{\DUrole{n}{postcodes}}}{}
\pysigstopsignatures
\sphinxAtStartPar
Get a frame of OS eastings and northings from a collection
of input postcodes.
\begin{quote}\begin{description}
\sphinxlineitem{Parameters}
\sphinxAtStartPar
\sphinxstyleliteralstrong{\sphinxupquote{postcodes}} (\sphinxstyleliteralemphasis{\sphinxupquote{sequence of strs}}) \textendash{} Sequence of postcodes.

\sphinxlineitem{Returns}
\sphinxAtStartPar
DataFrame containing only OSGB36 easthing and northing indexed
by the input postcodes. Invalid postcodes (i.e. not in the
input unlabelled postcodes file) return as NaN.

\sphinxlineitem{Return type}
\sphinxAtStartPar
pandas.DataFrame

\end{description}\end{quote}
\subsubsection*{Example}

\begin{sphinxVerbatim}[commandchars=\\\{\}]
\PYG{g+gp}{\PYGZgt{}\PYGZgt{}\PYGZgt{} }\PYG{n}{get\PYGZus{}easting\PYGZus{}northing}\PYG{p}{(}\PYG{p}{[}\PYG{l+s+s1}{\PYGZsq{}}\PYG{l+s+s1}{DE2 3DA}\PYG{l+s+s1}{\PYGZsq{}}\PYG{p}{,} \PYG{l+s+s1}{\PYGZsq{}}\PYG{l+s+s1}{LN5 7RW}\PYG{l+s+s1}{\PYGZsq{}}\PYG{p}{]}\PYG{p}{)}
\PYG{g+go}{           easting  northing}
\PYG{g+go}{postcode}
\PYG{g+go}{DE2 3DA   434011.0  330722.0}
\PYG{g+go}{LN5 7RW   497441.0  370798.0}
\end{sphinxVerbatim}

\begin{sphinxVerbatim}[commandchars=\\\{\}]
\PYG{g+gp}{\PYGZgt{}\PYGZgt{}\PYGZgt{} }\PYG{n}{get\PYGZus{}easting\PYGZus{}northing}\PYG{p}{(}\PYG{p}{[}\PYG{l+s+s1}{\PYGZsq{}}\PYG{l+s+s1}{XX1 2XX}\PYG{l+s+s1}{\PYGZsq{}}\PYG{p}{]}\PYG{p}{)}
\PYG{g+go}{          easting  northing}
\PYG{g+go}{postcode}
\PYG{g+go}{XX2 3XAA      NaN       NaN}
\end{sphinxVerbatim}

\end{fulllineitems}

\index{get\_flood\_class() (flood\_tool.Tool method)@\spxentry{get\_flood\_class()}\spxextra{flood\_tool.Tool method}}

\begin{fulllineitems}
\phantomsection\label{\detokenize{index:flood_tool.Tool.get_flood_class}}
\pysigstartsignatures
\pysiglinewithargsret{\sphinxbfcode{\sphinxupquote{get\_flood\_class}}}{\emph{\DUrole{n}{postcodes}}, \emph{\DUrole{n}{method}\DUrole{o}{=}\DUrole{default_value}{0}}}{}
\pysigstopsignatures
\sphinxAtStartPar
Generate series predicting flood probability classification
for a collection of poscodes.
\begin{quote}\begin{description}
\sphinxlineitem{Parameters}\begin{itemize}
\item {} 
\sphinxAtStartPar
\sphinxstyleliteralstrong{\sphinxupquote{postcodes}} (\sphinxstyleliteralemphasis{\sphinxupquote{sequence of strs}}) \textendash{} Sequence of postcodes.

\item {} 
\sphinxAtStartPar
\sphinxstyleliteralstrong{\sphinxupquote{method}} (\sphinxstyleliteralemphasis{\sphinxupquote{int}}\sphinxstyleliteralemphasis{\sphinxupquote{ (}}\sphinxstyleliteralemphasis{\sphinxupquote{optional}}\sphinxstyleliteralemphasis{\sphinxupquote{)}}) \textendash{} optionally specify (via a value in
self.get\_flood\_probability\_methods) the classification
method to be used.

\end{itemize}

\sphinxlineitem{Returns}
\sphinxAtStartPar
Series of flood risk classification labels indexed by postcodes.

\sphinxlineitem{Return type}
\sphinxAtStartPar
pandas.Series

\end{description}\end{quote}

\end{fulllineitems}

\index{get\_flood\_class\_methods() (flood\_tool.Tool static method)@\spxentry{get\_flood\_class\_methods()}\spxextra{flood\_tool.Tool static method}}

\begin{fulllineitems}
\phantomsection\label{\detokenize{index:flood_tool.Tool.get_flood_class_methods}}
\pysigstartsignatures
\pysiglinewithargsret{\sphinxbfcode{\sphinxupquote{static\DUrole{w}{  }}}\sphinxbfcode{\sphinxupquote{get\_flood\_class\_methods}}}{}{}
\pysigstopsignatures
\sphinxAtStartPar
Get a dictionary of available flood probablity classification methods.
\begin{quote}\begin{description}
\sphinxlineitem{Returns}
\sphinxAtStartPar
\begin{description}
\sphinxlineitem{Dictionary mapping classification method names (which have}
\sphinxAtStartPar
no inate meaning) on to an identifier to be passed to the
get\_flood\_probability method.

\end{description}


\sphinxlineitem{Return type}
\sphinxAtStartPar
dict

\end{description}\end{quote}

\end{fulllineitems}

\index{get\_flood\_class\_models() (flood\_tool.Tool static method)@\spxentry{get\_flood\_class\_models()}\spxextra{flood\_tool.Tool static method}}

\begin{fulllineitems}
\phantomsection\label{\detokenize{index:flood_tool.Tool.get_flood_class_models}}
\pysigstartsignatures
\pysiglinewithargsret{\sphinxbfcode{\sphinxupquote{static\DUrole{w}{  }}}\sphinxbfcode{\sphinxupquote{get\_flood\_class\_models}}}{}{}
\pysigstopsignatures\begin{quote}\begin{description}
\sphinxlineitem{Returns}
\sphinxAtStartPar
method to model

\sphinxlineitem{Return type}
\sphinxAtStartPar
Dict

\end{description}\end{quote}

\end{fulllineitems}

\index{get\_house\_price\_methods() (flood\_tool.Tool static method)@\spxentry{get\_house\_price\_methods()}\spxextra{flood\_tool.Tool static method}}

\begin{fulllineitems}
\phantomsection\label{\detokenize{index:flood_tool.Tool.get_house_price_methods}}
\pysigstartsignatures
\pysiglinewithargsret{\sphinxbfcode{\sphinxupquote{static\DUrole{w}{  }}}\sphinxbfcode{\sphinxupquote{get\_house\_price\_methods}}}{}{}
\pysigstopsignatures
\sphinxAtStartPar
Get a dictionary of available flood house price regression methods.
\begin{quote}\begin{description}
\sphinxlineitem{Returns}
\sphinxAtStartPar
\begin{description}
\sphinxlineitem{Dictionary mapping regression method names (which have}
\sphinxAtStartPar
no inate meaning) on to an identifier to be passed to the
get\_median\_house\_price\_estimate method.

\end{description}


\sphinxlineitem{Return type}
\sphinxAtStartPar
dict

\end{description}\end{quote}

\end{fulllineitems}

\index{get\_house\_price\_models() (flood\_tool.Tool static method)@\spxentry{get\_house\_price\_models()}\spxextra{flood\_tool.Tool static method}}

\begin{fulllineitems}
\phantomsection\label{\detokenize{index:flood_tool.Tool.get_house_price_models}}
\pysigstartsignatures
\pysiglinewithargsret{\sphinxbfcode{\sphinxupquote{static\DUrole{w}{  }}}\sphinxbfcode{\sphinxupquote{get\_house\_price\_models}}}{}{}
\pysigstopsignatures\begin{quote}\begin{description}
\sphinxlineitem{Returns}
\sphinxAtStartPar
method to model

\sphinxlineitem{Return type}
\sphinxAtStartPar
Dict

\end{description}\end{quote}

\end{fulllineitems}

\index{get\_lat\_long() (flood\_tool.Tool method)@\spxentry{get\_lat\_long()}\spxextra{flood\_tool.Tool method}}

\begin{fulllineitems}
\phantomsection\label{\detokenize{index:flood_tool.Tool.get_lat_long}}
\pysigstartsignatures
\pysiglinewithargsret{\sphinxbfcode{\sphinxupquote{get\_lat\_long}}}{\emph{\DUrole{n}{postcodes}}}{}
\pysigstopsignatures
\sphinxAtStartPar
Get a frame containing GPS latitude and longitude information for a
collection of of postcodes.
\begin{quote}\begin{description}
\sphinxlineitem{Parameters}
\sphinxAtStartPar
\sphinxstyleliteralstrong{\sphinxupquote{postcodes}} (\sphinxstyleliteralemphasis{\sphinxupquote{sequence of strs}}) \textendash{} Sequence of postcodes.

\sphinxlineitem{Returns}
\sphinxAtStartPar
DataFrame containing only WGS84 latitude and longitude pairs for
the input postcodes. Invalid postcodes (i.e. not in the
input unlabelled postcodes file) return as NAN.

\sphinxlineitem{Return type}
\sphinxAtStartPar
pandas.DataFrame

\end{description}\end{quote}
\subsubsection*{Example}

\begin{sphinxVerbatim}[commandchars=\\\{\}]
\PYG{g+gp}{\PYGZgt{}\PYGZgt{}\PYGZgt{} }\PYG{n}{get\PYGZus{}lat\PYGZus{}long}\PYG{p}{(}\PYG{p}{[}\PYG{l+s+s1}{\PYGZsq{}}\PYG{l+s+s1}{DE2 3DA}\PYG{l+s+s1}{\PYGZsq{}}\PYG{p}{,} \PYG{l+s+s1}{\PYGZsq{}}\PYG{l+s+s1}{LN5 7RW}\PYG{l+s+s1}{\PYGZsq{}}\PYG{p}{]}\PYG{p}{)}
\PYG{g+go}{           latitude  logitude}
\PYG{g+go}{postcode}
\PYG{g+go}{DE2 3DA   52.872902 \PYGZhy{}1.496148}
\PYG{g+go}{LN5 7RW   53.225300 \PYGZhy{}0.541897}
\end{sphinxVerbatim}

\begin{sphinxVerbatim}[commandchars=\\\{\}]
\PYG{g+gp}{\PYGZgt{}\PYGZgt{}\PYGZgt{} }\PYG{n}{get\PYGZus{}lat\PYGZus{}long}\PYG{p}{(}\PYG{p}{[}\PYG{l+s+s1}{\PYGZsq{}}\PYG{l+s+s1}{DE2 3DA}\PYG{l+s+s1}{\PYGZsq{}}\PYG{p}{,} \PYG{l+s+s1}{\PYGZsq{}}\PYG{l+s+s1}{LN5 7RW}\PYG{l+s+s1}{\PYGZsq{}}\PYG{p}{]}\PYG{p}{)}
\PYG{g+go}{          latitude  longitude}
\PYG{g+go}{postcode}
\PYG{g+go}{XX2 3XAA       NaN      NaN}
\end{sphinxVerbatim}

\end{fulllineitems}

\index{get\_median\_house\_price\_estimate() (flood\_tool.Tool method)@\spxentry{get\_median\_house\_price\_estimate()}\spxextra{flood\_tool.Tool method}}

\begin{fulllineitems}
\phantomsection\label{\detokenize{index:flood_tool.Tool.get_median_house_price_estimate}}
\pysigstartsignatures
\pysiglinewithargsret{\sphinxbfcode{\sphinxupquote{get\_median\_house\_price\_estimate}}}{\emph{\DUrole{n}{postcodes}}, \emph{\DUrole{n}{method}\DUrole{o}{=}\DUrole{default_value}{0}}}{}
\pysigstopsignatures
\sphinxAtStartPar
Generate series predicting median house price for a collection
of poscodes.
\begin{quote}\begin{description}
\sphinxlineitem{Parameters}\begin{itemize}
\item {} 
\sphinxAtStartPar
\sphinxstyleliteralstrong{\sphinxupquote{postcodes}} (\sphinxstyleliteralemphasis{\sphinxupquote{sequence of strs}}) \textendash{} Sequence of postcodes.

\item {} 
\sphinxAtStartPar
\sphinxstyleliteralstrong{\sphinxupquote{method}} (\sphinxstyleliteralemphasis{\sphinxupquote{int}}\sphinxstyleliteralemphasis{\sphinxupquote{ (}}\sphinxstyleliteralemphasis{\sphinxupquote{optional}}\sphinxstyleliteralemphasis{\sphinxupquote{)}}) \textendash{} optionally specify (via a value in
self.get\_house\_price\_methods) the regression
method to be used.

\end{itemize}

\sphinxlineitem{Returns}
\sphinxAtStartPar
Series of median house price estimates indexed by postcodes.

\sphinxlineitem{Return type}
\sphinxAtStartPar
pandas.Series

\end{description}\end{quote}

\end{fulllineitems}

\index{get\_postcode\_from\_sector() (flood\_tool.Tool method)@\spxentry{get\_postcode\_from\_sector()}\spxextra{flood\_tool.Tool method}}

\begin{fulllineitems}
\phantomsection\label{\detokenize{index:flood_tool.Tool.get_postcode_from_sector}}
\pysigstartsignatures
\pysiglinewithargsret{\sphinxbfcode{\sphinxupquote{get\_postcode\_from\_sector}}}{\emph{\DUrole{n}{sectors}}}{}
\pysigstopsignatures
\sphinxAtStartPar
Get a frame of postcodes from a collection
of input sectors.
\begin{quote}\begin{description}
\sphinxlineitem{Parameters}
\sphinxAtStartPar
\sphinxstyleliteralstrong{\sphinxupquote{postcodes}} (\sphinxstyleliteralemphasis{\sphinxupquote{sequence of strs}}) \textendash{} Sequence of sectors.

\sphinxlineitem{Returns}
\sphinxAtStartPar
DataFrame containing only OSGB36 easthing and northing indexed
by the input sectors. Invalid sectors (i.e. not in the
input unlabelled postcodes file) return as NaN.

\sphinxlineitem{Return type}
\sphinxAtStartPar
pandas.DataFrame

\end{description}\end{quote}

\end{fulllineitems}

\index{get\_total\_value() (flood\_tool.Tool method)@\spxentry{get\_total\_value()}\spxextra{flood\_tool.Tool method}}

\begin{fulllineitems}
\phantomsection\label{\detokenize{index:flood_tool.Tool.get_total_value}}
\pysigstartsignatures
\pysiglinewithargsret{\sphinxbfcode{\sphinxupquote{get\_total\_value}}}{\emph{\DUrole{n}{locations}}}{}
\pysigstopsignatures
\sphinxAtStartPar
Return a series of estimates of the total property values
of a collection of postcode units or sectors.
\begin{quote}\begin{description}
\sphinxlineitem{Parameters}
\sphinxAtStartPar
\sphinxstyleliteralstrong{\sphinxupquote{locations}} (\sphinxstyleliteralemphasis{\sphinxupquote{sequence of strs}}) \textendash{} Sequence of postcode units or sectors

\sphinxlineitem{Returns}
\sphinxAtStartPar
Series of total property value estimates indexed by locations.

\sphinxlineitem{Return type}
\sphinxAtStartPar
pandas.Series

\end{description}\end{quote}

\end{fulllineitems}


\end{fulllineitems}

\index{get\_closest\_station\_ref\_by\_type\_from\_lat\_lng() (in module flood\_tool)@\spxentry{get\_closest\_station\_ref\_by\_type\_from\_lat\_lng()}\spxextra{in module flood\_tool}}

\begin{fulllineitems}
\phantomsection\label{\detokenize{index:flood_tool.get_closest_station_ref_by_type_from_lat_lng}}
\pysigstartsignatures
\pysiglinewithargsret{\sphinxcode{\sphinxupquote{flood\_tool.}}\sphinxbfcode{\sphinxupquote{get\_closest\_station\_ref\_by\_type\_from\_lat\_lng}}}{\emph{\DUrole{n}{lat}}, \emph{\DUrole{n}{lng}}, \emph{\DUrole{n}{s\_type}\DUrole{o}{=}\DUrole{default_value}{\textquotesingle{}rainfall\textquotesingle{}}}}{}
\pysigstopsignatures
\sphinxAtStartPar
Return stationReference (rainfall or level) where the station is
closest to the given lat, lng.
\begin{quote}\begin{description}
\sphinxlineitem{Parameters}\begin{itemize}
\item {} 
\sphinxAtStartPar
\sphinxstyleliteralstrong{\sphinxupquote{lat}} (\sphinxstyleliteralemphasis{\sphinxupquote{float}}) \textendash{} latitude

\item {} 
\sphinxAtStartPar
\sphinxstyleliteralstrong{\sphinxupquote{lng}} (\sphinxstyleliteralemphasis{\sphinxupquote{float}}) \textendash{} longitude

\item {} 
\sphinxAtStartPar
\sphinxstyleliteralstrong{\sphinxupquote{type}} (\sphinxstyleliteralemphasis{\sphinxupquote{str}}) \textendash{} rainfall / level station

\end{itemize}

\sphinxlineitem{Returns}
\sphinxAtStartPar
\begin{itemize}
\item {} 
\sphinxAtStartPar
\sphinxstylestrong{ref} (\sphinxstyleemphasis{str})

\item {} 
\sphinxAtStartPar
\sphinxstyleemphasis{stationReference}

\end{itemize}


\end{description}\end{quote}
\subsubsection*{Example}

\begin{sphinxVerbatim}[commandchars=\\\{\}]
\PYG{g+gp}{\PYGZgt{}\PYGZgt{}\PYGZgt{} }\PYG{n}{ref} \PYG{o}{=} \PYG{n}{get\PYGZus{}closest\PYGZus{}station\PYGZus{}ref\PYGZus{}from\PYGZus{}lat\PYGZus{}lng}\PYG{p}{(}\PYG{l+m+mf}{52.872902}\PYG{p}{,} \PYG{o}{\PYGZhy{}}\PYG{l+m+mf}{1.496148}\PYG{p}{)}
\end{sphinxVerbatim}

\end{fulllineitems}

\index{get\_closest\_station\_ref\_from\_lat\_lng() (in module flood\_tool)@\spxentry{get\_closest\_station\_ref\_from\_lat\_lng()}\spxextra{in module flood\_tool}}

\begin{fulllineitems}
\phantomsection\label{\detokenize{index:flood_tool.get_closest_station_ref_from_lat_lng}}
\pysigstartsignatures
\pysiglinewithargsret{\sphinxcode{\sphinxupquote{flood\_tool.}}\sphinxbfcode{\sphinxupquote{get\_closest\_station\_ref\_from\_lat\_lng}}}{\emph{\DUrole{n}{lat}}, \emph{\DUrole{n}{lng}}}{}
\pysigstopsignatures
\sphinxAtStartPar
Return stationReference where the station is
closest to the given lat, lng.
\begin{quote}\begin{description}
\sphinxlineitem{Parameters}\begin{itemize}
\item {} 
\sphinxAtStartPar
\sphinxstyleliteralstrong{\sphinxupquote{lat}} (\sphinxstyleliteralemphasis{\sphinxupquote{float}}) \textendash{} latitude

\item {} 
\sphinxAtStartPar
\sphinxstyleliteralstrong{\sphinxupquote{lng}} (\sphinxstyleliteralemphasis{\sphinxupquote{float}}) \textendash{} longitude

\end{itemize}

\sphinxlineitem{Returns}
\sphinxAtStartPar
\begin{itemize}
\item {} 
\sphinxAtStartPar
\sphinxstylestrong{ref} (\sphinxstyleemphasis{str})

\item {} 
\sphinxAtStartPar
\sphinxstyleemphasis{stationReference}

\end{itemize}


\end{description}\end{quote}
\subsubsection*{Example}

\begin{sphinxVerbatim}[commandchars=\\\{\}]
\PYG{g+gp}{\PYGZgt{}\PYGZgt{}\PYGZgt{} }\PYG{n}{ref} \PYG{o}{=} \PYG{n}{get\PYGZus{}closest\PYGZus{}station\PYGZus{}ref\PYGZus{}from\PYGZus{}lat\PYGZus{}lng}\PYG{p}{(}\PYG{l+m+mf}{52.872902}\PYG{p}{,} \PYG{o}{\PYGZhy{}}\PYG{l+m+mf}{1.496148}\PYG{p}{)}
\end{sphinxVerbatim}

\end{fulllineitems}

\index{get\_easting\_northing\_from\_gps\_lat\_long() (in module flood\_tool)@\spxentry{get\_easting\_northing\_from\_gps\_lat\_long()}\spxextra{in module flood\_tool}}

\begin{fulllineitems}
\phantomsection\label{\detokenize{index:flood_tool.get_easting_northing_from_gps_lat_long}}
\pysigstartsignatures
\pysiglinewithargsret{\sphinxcode{\sphinxupquote{flood\_tool.}}\sphinxbfcode{\sphinxupquote{get\_easting\_northing\_from\_gps\_lat\_long}}}{\emph{\DUrole{n}{phi}}, \emph{\DUrole{n}{lam}}, \emph{\DUrole{n}{rads}\DUrole{o}{=}\DUrole{default_value}{False}}}{}
\pysigstopsignatures
\sphinxAtStartPar
Get OSGB36 easting/northing from GPS latitude and
longitude pairs.
\begin{quote}\begin{description}
\sphinxlineitem{Parameters}\begin{itemize}
\item {} 
\sphinxAtStartPar
\sphinxstyleliteralstrong{\sphinxupquote{phi}} (\sphinxstyleliteralemphasis{\sphinxupquote{float/arraylike}}) \textendash{} GPS (i.e. WGS84 datum) latitude value(s)

\item {} 
\sphinxAtStartPar
\sphinxstyleliteralstrong{\sphinxupquote{lam}} (\sphinxstyleliteralemphasis{\sphinxupquote{float/arraylike}}) \textendash{} GPS (i.e. WGS84 datum) longitude value(s).

\item {} 
\sphinxAtStartPar
\sphinxstyleliteralstrong{\sphinxupquote{rads}} (\sphinxstyleliteralemphasis{\sphinxupquote{bool}}\sphinxstyleliteralemphasis{\sphinxupquote{ (}}\sphinxstyleliteralemphasis{\sphinxupquote{optional}}\sphinxstyleliteralemphasis{\sphinxupquote{)}}) \textendash{} If true, specifies input is is radians.

\end{itemize}

\sphinxlineitem{Returns}
\sphinxAtStartPar
\begin{itemize}
\item {} 
\sphinxAtStartPar
\sphinxstyleemphasis{numpy.ndarray} \textendash{} Easting values (in m)

\item {} 
\sphinxAtStartPar
\sphinxstyleemphasis{numpy.ndarray} \textendash{} Northing values (in m)

\end{itemize}


\end{description}\end{quote}
\subsubsection*{Examples}

\begin{sphinxVerbatim}[commandchars=\\\{\}]
\PYG{g+gp}{\PYGZgt{}\PYGZgt{}\PYGZgt{} }\PYG{n}{get\PYGZus{}easting\PYGZus{}northing\PYGZus{}from\PYGZus{}gps\PYGZus{}lat\PYGZus{}long}\PYG{p}{(}\PYG{p}{[}\PYG{l+m+mf}{55.5}\PYG{p}{]}\PYG{p}{,} \PYG{p}{[}\PYG{o}{\PYGZhy{}}\PYG{l+m+mf}{1.54}\PYG{p}{]}\PYG{p}{)}
\PYG{g+go}{(array([429157.0]), array([623009]))}
\end{sphinxVerbatim}
\subsubsection*{References}

\sphinxAtStartPar
Based on the formulas in “A guide to coordinate systems in Great Britain”.

\sphinxAtStartPar
See also \sphinxurl{https://webapps.bgs.ac.uk/data/webservices/convertForm.cfm}

\end{fulllineitems}

\index{get\_gps\_lat\_long\_from\_easting\_northing() (in module flood\_tool)@\spxentry{get\_gps\_lat\_long\_from\_easting\_northing()}\spxextra{in module flood\_tool}}

\begin{fulllineitems}
\phantomsection\label{\detokenize{index:flood_tool.get_gps_lat_long_from_easting_northing}}
\pysigstartsignatures
\pysiglinewithargsret{\sphinxcode{\sphinxupquote{flood\_tool.}}\sphinxbfcode{\sphinxupquote{get\_gps\_lat\_long\_from\_easting\_northing}}}{\emph{\DUrole{n}{east}}, \emph{\DUrole{n}{north}}, \emph{\DUrole{n}{rads}\DUrole{o}{=}\DUrole{default_value}{False}}, \emph{\DUrole{n}{dms}\DUrole{o}{=}\DUrole{default_value}{False}}}{}
\pysigstopsignatures
\sphinxAtStartPar
Get OSGB36 easting/northing from GPS latitude and
longitude pairs.
\begin{quote}\begin{description}
\sphinxlineitem{Parameters}\begin{itemize}
\item {} 
\sphinxAtStartPar
\sphinxstyleliteralstrong{\sphinxupquote{east}} (\sphinxstyleliteralemphasis{\sphinxupquote{float/arraylike}}) \textendash{} OSGB36 easting value(s) (in m).

\item {} 
\sphinxAtStartPar
\sphinxstyleliteralstrong{\sphinxupquote{north}} (\sphinxstyleliteralemphasis{\sphinxupquote{float/arraylike}}) \textendash{} OSGB36 easting value(s) (in m).

\item {} 
\sphinxAtStartPar
\sphinxstyleliteralstrong{\sphinxupquote{rads}} (\sphinxstyleliteralemphasis{\sphinxupquote{bool}}\sphinxstyleliteralemphasis{\sphinxupquote{ (}}\sphinxstyleliteralemphasis{\sphinxupquote{optional}}\sphinxstyleliteralemphasis{\sphinxupquote{)}}) \textendash{} If true, specifies ouput is is radians.

\item {} 
\sphinxAtStartPar
\sphinxstyleliteralstrong{\sphinxupquote{dms}} (\sphinxstyleliteralemphasis{\sphinxupquote{bool}}\sphinxstyleliteralemphasis{\sphinxupquote{ (}}\sphinxstyleliteralemphasis{\sphinxupquote{optional}}\sphinxstyleliteralemphasis{\sphinxupquote{)}}) \textendash{} If true, output is in degrees/minutes/seconds. Incompatible
with rads option.

\end{itemize}

\sphinxlineitem{Returns}
\sphinxAtStartPar
\begin{itemize}
\item {} 
\sphinxAtStartPar
\sphinxstyleemphasis{numpy.ndarray} \textendash{} GPS (i.e. WGS84 datum) latitude value(s).

\item {} 
\sphinxAtStartPar
\sphinxstyleemphasis{numpy.ndarray} \textendash{} GPS (i.e. WGS84 datum) longitude value(s).

\end{itemize}


\end{description}\end{quote}
\subsubsection*{Examples}

\begin{sphinxVerbatim}[commandchars=\\\{\}]
\PYG{g+gp}{\PYGZgt{}\PYGZgt{}\PYGZgt{} }\PYG{n}{get\PYGZus{}gps\PYGZus{}lat\PYGZus{}long\PYGZus{}from\PYGZus{}easting\PYGZus{}northing}\PYG{p}{(}\PYG{p}{[}\PYG{l+m+mi}{429157}\PYG{p}{]}\PYG{p}{,} \PYG{p}{[}\PYG{l+m+mi}{623009}\PYG{p}{]}\PYG{p}{)}
\PYG{g+go}{(array([55.5]), array([\PYGZhy{}1.540008]))}
\end{sphinxVerbatim}
\subsubsection*{References}

\sphinxAtStartPar
Based on the formulas in “A guide to coordinate systems in Great Britain”.

\sphinxAtStartPar
See also \sphinxurl{https://webapps.bgs.ac.uk/data/webservices/convertForm.cfm}

\end{fulllineitems}

\index{get\_grouped\_reading() (in module flood\_tool)@\spxentry{get\_grouped\_reading()}\spxextra{in module flood\_tool}}

\begin{fulllineitems}
\phantomsection\label{\detokenize{index:flood_tool.get_grouped_reading}}
\pysigstartsignatures
\pysiglinewithargsret{\sphinxcode{\sphinxupquote{flood\_tool.}}\sphinxbfcode{\sphinxupquote{get\_grouped\_reading}}}{\emph{\DUrole{n}{file}}, \emph{\DUrole{n}{parameter}}}{}
\pysigstopsignatures
\sphinxAtStartPar
Get sum of rainfall / max water level indexed by station reference
\subsubsection*{Example}

\begin{sphinxVerbatim}[commandchars=\\\{\}]
\PYG{g+gp}{\PYGZgt{}\PYGZgt{}\PYGZgt{} }\PYG{n}{get\PYGZus{}grouped\PYGZus{}reading}\PYG{p}{(}\PYG{l+s+s2}{\PYGZdq{}}\PYG{l+s+s2}{wet\PYGZus{}day.csv}\PYG{l+s+s2}{\PYGZdq{}}\PYG{p}{,} \PYG{l+s+s2}{\PYGZdq{}}\PYG{l+s+s2}{rainfall}\PYG{l+s+s2}{\PYGZdq{}}\PYG{p}{)}
\end{sphinxVerbatim}
\begin{quote}

\sphinxAtStartPar
value
\end{quote}

\sphinxAtStartPar
stationReference
000008              0.0
000028              0.2
000075TP            0.0

\end{fulllineitems}

\index{get\_rainfall\_classifier\_from\_lat\_lng() (in module flood\_tool)@\spxentry{get\_rainfall\_classifier\_from\_lat\_lng()}\spxextra{in module flood\_tool}}

\begin{fulllineitems}
\phantomsection\label{\detokenize{index:flood_tool.get_rainfall_classifier_from_lat_lng}}
\pysigstartsignatures
\pysiglinewithargsret{\sphinxcode{\sphinxupquote{flood\_tool.}}\sphinxbfcode{\sphinxupquote{get\_rainfall\_classifier\_from\_lat\_lng}}}{\emph{\DUrole{n}{filename}}, \emph{\DUrole{n}{lat}}, \emph{\DUrole{n}{lng}}}{}
\pysigstopsignatures
\sphinxAtStartPar
Return rainfall class of the station
that is the closest to the given lat, lng.
\begin{quote}\begin{description}
\sphinxlineitem{Parameters}\begin{itemize}
\item {} 
\sphinxAtStartPar
\sphinxstyleliteralstrong{\sphinxupquote{lat}} (\sphinxstyleliteralemphasis{\sphinxupquote{float}}) \textendash{} latitude

\item {} 
\sphinxAtStartPar
\sphinxstyleliteralstrong{\sphinxupquote{lng}} (\sphinxstyleliteralemphasis{\sphinxupquote{float}}) \textendash{} longitude

\end{itemize}

\sphinxlineitem{Returns}
\sphinxAtStartPar
\begin{itemize}
\item {} 
\sphinxAtStartPar
\sphinxstylestrong{rain\_class} (\sphinxstyleemphasis{str})

\item {} 
\sphinxAtStartPar
\sphinxstylestrong{rainfall(mm) class} (\sphinxstyleemphasis{no rain, slight, moderate, heavy, violent})

\end{itemize}


\end{description}\end{quote}

\end{fulllineitems}

\index{get\_rainfall\_riverlevel\_from\_lat\_lng() (in module flood\_tool)@\spxentry{get\_rainfall\_riverlevel\_from\_lat\_lng()}\spxextra{in module flood\_tool}}

\begin{fulllineitems}
\phantomsection\label{\detokenize{index:flood_tool.get_rainfall_riverlevel_from_lat_lng}}
\pysigstartsignatures
\pysiglinewithargsret{\sphinxcode{\sphinxupquote{flood\_tool.}}\sphinxbfcode{\sphinxupquote{get\_rainfall\_riverlevel\_from\_lat\_lng}}}{\emph{\DUrole{n}{filename}}, \emph{\DUrole{n}{lat}}, \emph{\DUrole{n}{lng}}}{}
\pysigstopsignatures
\sphinxAtStartPar
Return total rainfall and max river level where given the
latitude and longitude in a day.
\begin{quote}\begin{description}
\sphinxlineitem{Parameters}\begin{itemize}
\item {} 
\sphinxAtStartPar
\sphinxstyleliteralstrong{\sphinxupquote{filename}} (\sphinxstyleliteralemphasis{\sphinxupquote{str}}) \textendash{} typical\_day or wet\_day

\item {} 
\sphinxAtStartPar
\sphinxstyleliteralstrong{\sphinxupquote{lat}} (\sphinxstyleliteralemphasis{\sphinxupquote{float}}) \textendash{} latitude

\item {} 
\sphinxAtStartPar
\sphinxstyleliteralstrong{\sphinxupquote{lng}} (\sphinxstyleliteralemphasis{\sphinxupquote{float}}) \textendash{} longitude

\end{itemize}

\sphinxlineitem{Returns}
\sphinxAtStartPar
\begin{itemize}
\item {} 
\sphinxAtStartPar
\sphinxstylestrong{total\_rainfall} (\sphinxstyleemphasis{float}) \textendash{} total rainfall of the day (unit: mm)

\item {} 
\sphinxAtStartPar
\sphinxstylestrong{max\_level} (\sphinxstyleemphasis{float}) \textendash{} max of the river level of the day (unit: mASD)

\end{itemize}


\end{description}\end{quote}
\subsubsection*{Example}

\begin{sphinxVerbatim}[commandchars=\\\{\}]
\PYG{g+gp}{\PYGZgt{}\PYGZgt{}\PYGZgt{} }\PYG{n}{rainfall}\PYG{p}{,} \PYG{n}{level} \PYG{o}{=} \PYG{n}{get\PYGZus{}rainfall\PYGZus{}riverlevel\PYGZus{}from\PYGZus{}lat\PYGZus{}lng}\PYG{p}{(}\PYG{l+s+s1}{\PYGZsq{}}\PYG{l+s+s1}{flood\PYGZus{}tool/resources/wet\PYGZus{}day.csv}\PYG{l+s+s1}{\PYGZsq{}}\PYG{p}{,} \PYG{l+m+mf}{52.872902}\PYG{p}{,} \PYG{o}{\PYGZhy{}}\PYG{l+m+mf}{1.496148}\PYG{p}{)}
\end{sphinxVerbatim}

\end{fulllineitems}

\index{get\_station\_data() (in module flood\_tool)@\spxentry{get\_station\_data()}\spxextra{in module flood\_tool}}

\begin{fulllineitems}
\phantomsection\label{\detokenize{index:flood_tool.get_station_data}}
\pysigstartsignatures
\pysiglinewithargsret{\sphinxcode{\sphinxupquote{flood\_tool.}}\sphinxbfcode{\sphinxupquote{get\_station\_data}}}{\emph{\DUrole{n}{filename}}, \emph{\DUrole{n}{station\_reference}}}{}
\pysigstopsignatures
\sphinxAtStartPar
Return readings for a specified recording station from .csv file.
\begin{quote}\begin{description}
\sphinxlineitem{Parameters}\begin{itemize}
\item {} 
\sphinxAtStartPar
\sphinxstyleliteralstrong{\sphinxupquote{filename}} (\sphinxstyleliteralemphasis{\sphinxupquote{str}}) \textendash{} filename to read

\item {} 
\sphinxAtStartPar
\sphinxstyleliteralstrong{\sphinxupquote{station\_reference}} \textendash{} station\_reference to return.

\item {} 
\sphinxAtStartPar
\sphinxstyleliteralstrong{\sphinxupquote{get\_station\_data}}\sphinxstyleliteralstrong{\sphinxupquote{(}}\sphinxstyleliteralstrong{\sphinxupquote{\textquotesingle{}flood\_tool/resources/wet\_day.csv}}\sphinxstyleliteralstrong{\sphinxupquote{)}} (\sphinxstyleliteralemphasis{\sphinxupquote{\textgreater{}\textgreater{}\textgreater{} data =}}) \textendash{} 

\end{itemize}

\end{description}\end{quote}

\end{fulllineitems}

\index{get\_station\_reading() (in module flood\_tool)@\spxentry{get\_station\_reading()}\spxextra{in module flood\_tool}}

\begin{fulllineitems}
\phantomsection\label{\detokenize{index:flood_tool.get_station_reading}}
\pysigstartsignatures
\pysiglinewithargsret{\sphinxcode{\sphinxupquote{flood\_tool.}}\sphinxbfcode{\sphinxupquote{get\_station\_reading}}}{\emph{\DUrole{n}{file}}, \emph{\DUrole{n}{parameter}}, \emph{\DUrole{n}{station\_ref}}}{}
\pysigstopsignatures
\sphinxAtStartPar
Get rainfall or water level data for a list of staions.
Return DataFrame indexed by dateTime, with station name as the column names.
\begin{quote}\begin{description}
\sphinxlineitem{Parameters}\begin{itemize}
\item {} 
\sphinxAtStartPar
\sphinxstyleliteralstrong{\sphinxupquote{file}} (\sphinxstyleliteralemphasis{\sphinxupquote{str}}) \textendash{} filename / path of the rainfall \& water level reading file

\item {} 
\sphinxAtStartPar
\sphinxstyleliteralstrong{\sphinxupquote{parameter}} (\sphinxstyleliteralemphasis{\sphinxupquote{str}}) \textendash{} rainfall or level

\item {} 
\sphinxAtStartPar
\sphinxstyleliteralstrong{\sphinxupquote{station\_ref}} (\sphinxstyleliteralemphasis{\sphinxupquote{list}}) \textendash{} list of station\_ref

\end{itemize}

\end{description}\end{quote}
\subsubsection*{Example}

\begin{sphinxVerbatim}[commandchars=\\\{\}]
\PYG{g+gp}{\PYGZgt{}\PYGZgt{}\PYGZgt{} }\PYG{n}{get\PYGZus{}station\PYGZus{}reading}\PYG{p}{(}\PYG{p}{[}\PYG{l+s+s1}{\PYGZsq{}}\PYG{l+s+s1}{4761}\PYG{l+s+s1}{\PYGZsq{}}\PYG{p}{,} \PYG{l+s+s1}{\PYGZsq{}}\PYG{l+s+s1}{E2694}\PYG{l+s+s1}{\PYGZsq{}}\PYG{p}{,} \PYG{l+s+s1}{\PYGZsq{}}\PYG{l+s+s1}{E4823}\PYG{l+s+s1}{\PYGZsq{}}\PYG{p}{]}\PYG{p}{)}
\PYG{g+go}{                 dateTime   4761  E2694  E4823}
\PYG{g+go}{0    2021\PYGZhy{}05\PYGZhy{}07T00:00:00Z  0.069  0.056    NaN}
\PYG{g+go}{1    2021\PYGZhy{}05\PYGZhy{}07T00:00:01Z    NaN    NaN  0.325}
\end{sphinxVerbatim}

\end{fulllineitems}

\index{plot\_24h\_rainfall() (in module flood\_tool)@\spxentry{plot\_24h\_rainfall()}\spxextra{in module flood\_tool}}

\begin{fulllineitems}
\phantomsection\label{\detokenize{index:flood_tool.plot_24h_rainfall}}
\pysigstartsignatures
\pysiglinewithargsret{\sphinxcode{\sphinxupquote{flood\_tool.}}\sphinxbfcode{\sphinxupquote{plot\_24h\_rainfall}}}{\emph{\DUrole{n}{postcodes}}}{}
\pysigstopsignatures
\sphinxAtStartPar
Plot rainfall level heatmap for the nearest station of the specific postcodes with a
24 hour timeline, rainfall level data acquired from wet\_day.csv
\begin{quote}\begin{description}
\sphinxlineitem{Parameters}
\sphinxAtStartPar
\sphinxstyleliteralstrong{\sphinxupquote{postcode}} (\sphinxstyleliteralemphasis{\sphinxupquote{sequence of strs}}) \textendash{} Sequence of postcodes.

\sphinxlineitem{Returns}
\sphinxAtStartPar
\begin{itemize}
\item {} 
\sphinxAtStartPar
\sphinxstyleemphasis{Animated Folium map object}

\item {} 
\sphinxAtStartPar
\sphinxstyleemphasis{showing daily variation for the rainlevel at nearest station of the specific postcodes}

\end{itemize}


\end{description}\end{quote}
\subsubsection*{Examples}

\begin{sphinxVerbatim}[commandchars=\\\{\}]
\PYG{g+gp}{\PYGZgt{}\PYGZgt{}\PYGZgt{} }\PYG{n+nb}{map} \PYG{o}{=} \PYG{n}{plot\PYGZus{}24h\PYGZus{}rainfall}\PYG{p}{(}\PYG{n}{postcodes}\PYG{p}{)}
\end{sphinxVerbatim}

\end{fulllineitems}

\index{plot\_circle() (in module flood\_tool)@\spxentry{plot\_circle()}\spxextra{in module flood\_tool}}

\begin{fulllineitems}
\phantomsection\label{\detokenize{index:flood_tool.plot_circle}}
\pysigstartsignatures
\pysiglinewithargsret{\sphinxcode{\sphinxupquote{flood\_tool.}}\sphinxbfcode{\sphinxupquote{plot\_circle}}}{\emph{\DUrole{n}{lat}}, \emph{\DUrole{n}{lon}}, \emph{\DUrole{n}{radius}}, \emph{\DUrole{n}{map}\DUrole{o}{=}\DUrole{default_value}{None}}, \emph{\DUrole{o}{**}\DUrole{n}{kwargs}}}{}
\pysigstopsignatures
\sphinxAtStartPar
Plot a circle on a map (creating a new folium map instance if necessary).
\begin{quote}\begin{description}
\sphinxlineitem{Parameters}\begin{itemize}
\item {} 
\sphinxAtStartPar
\sphinxstyleliteralstrong{\sphinxupquote{lat}} (\sphinxstyleliteralemphasis{\sphinxupquote{float}}) \textendash{} latitude of circle to plot (degrees)

\item {} 
\sphinxAtStartPar
\sphinxstyleliteralstrong{\sphinxupquote{lon}} (\sphinxstyleliteralemphasis{\sphinxupquote{float}}) \textendash{} longitude of circle to plot (degrees)

\item {} 
\sphinxAtStartPar
\sphinxstyleliteralstrong{\sphinxupquote{radius}} (\sphinxstyleliteralemphasis{\sphinxupquote{float}}) \textendash{} radius of circle to plot (m)

\item {} 
\sphinxAtStartPar
\sphinxstyleliteralstrong{\sphinxupquote{map}} (\sphinxstyleliteralemphasis{\sphinxupquote{folium.Map}}) \textendash{} existing map object

\end{itemize}

\sphinxlineitem{Return type}
\sphinxAtStartPar
Folium map object

\end{description}\end{quote}
\subsubsection*{Examples}

\begin{sphinxVerbatim}[commandchars=\\\{\}]
\PYG{g+gp}{\PYGZgt{}\PYGZgt{}\PYGZgt{} }\PYG{k+kn}{import} \PYG{n+nn}{folium}
\PYG{g+gp}{\PYGZgt{}\PYGZgt{}\PYGZgt{} }\PYG{n}{armageddon}\PYG{o}{.}\PYG{n}{plot\PYGZus{}circle}\PYG{p}{(}\PYG{l+m+mf}{52.79}\PYG{p}{,} \PYG{o}{\PYGZhy{}}\PYG{l+m+mf}{2.95}\PYG{p}{,} \PYG{l+m+mf}{1e3}\PYG{p}{,} \PYG{n+nb}{map}\PYG{o}{=}\PYG{k+kc}{None}\PYG{p}{)}
\end{sphinxVerbatim}

\end{fulllineitems}

\index{plot\_flood\_prob\_level\_with\_given\_postcode() (in module flood\_tool)@\spxentry{plot\_flood\_prob\_level\_with\_given\_postcode()}\spxextra{in module flood\_tool}}

\begin{fulllineitems}
\phantomsection\label{\detokenize{index:flood_tool.plot_flood_prob_level_with_given_postcode}}
\pysigstartsignatures
\pysiglinewithargsret{\sphinxcode{\sphinxupquote{flood\_tool.}}\sphinxbfcode{\sphinxupquote{plot\_flood\_prob\_level\_with\_given\_postcode}}}{\emph{\DUrole{n}{postcodes}}}{}
\pysigstopsignatures
\sphinxAtStartPar
Plot flood probability map from given list of postcodes.
\begin{quote}\begin{description}
\sphinxlineitem{Parameters}
\sphinxAtStartPar
\sphinxstyleliteralstrong{\sphinxupquote{postcode}} (\sphinxstyleliteralemphasis{\sphinxupquote{sequence of strs}}) \textendash{} Sequence of postcodes.

\sphinxlineitem{Returns}
\sphinxAtStartPar
Different colours of circle markers show
different flood probabilities.

\sphinxlineitem{Return type}
\sphinxAtStartPar
Folium map object

\end{description}\end{quote}
\subsubsection*{Examples}

\begin{sphinxVerbatim}[commandchars=\\\{\}]
\PYG{g+gp}{\PYGZgt{}\PYGZgt{}\PYGZgt{} }\PYG{n}{mapping}\PYG{o}{.}\PYG{n}{plot\PYGZus{}flood\PYGZus{}prob\PYGZus{}level\PYGZus{}with\PYGZus{}given\PYGZus{}postcode}\PYG{p}{(}\PYG{p}{[}\PYG{l+s+s1}{\PYGZsq{}}\PYG{l+s+s1}{HP27 0BF}\PYG{l+s+s1}{\PYGZsq{}}\PYG{p}{,} \PYG{l+s+s1}{\PYGZsq{}}\PYG{l+s+s1}{W5 2BX}\PYG{l+s+s1}{\PYGZsq{}}\PYG{p}{]}\PYG{p}{)}
\end{sphinxVerbatim}

\end{fulllineitems}

\index{plot\_flood\_prob\_sampled() (in module flood\_tool)@\spxentry{plot\_flood\_prob\_sampled()}\spxextra{in module flood\_tool}}

\begin{fulllineitems}
\phantomsection\label{\detokenize{index:flood_tool.plot_flood_prob_sampled}}
\pysigstartsignatures
\pysiglinewithargsret{\sphinxcode{\sphinxupquote{flood\_tool.}}\sphinxbfcode{\sphinxupquote{plot\_flood\_prob\_sampled}}}{}{}
\pysigstopsignatures
\sphinxAtStartPar
Plot flood probability map from given sampled csv
\begin{quote}\begin{description}
\sphinxlineitem{Return type}
\sphinxAtStartPar
Folium map object

\end{description}\end{quote}
\subsubsection*{Examples}

\begin{sphinxVerbatim}[commandchars=\\\{\}]
\PYG{g+gp}{\PYGZgt{}\PYGZgt{}\PYGZgt{} }\PYG{n+nb}{map} \PYG{o}{=} \PYG{n}{plot\PYGZus{}flood\PYGZus{}prob\PYGZus{}sampled}\PYG{p}{(}\PYG{p}{)}
\end{sphinxVerbatim}

\end{fulllineitems}

\index{plot\_heatmap() (in module flood\_tool)@\spxentry{plot\_heatmap()}\spxextra{in module flood\_tool}}

\begin{fulllineitems}
\phantomsection\label{\detokenize{index:flood_tool.plot_heatmap}}
\pysigstartsignatures
\pysiglinewithargsret{\sphinxcode{\sphinxupquote{flood\_tool.}}\sphinxbfcode{\sphinxupquote{plot\_heatmap}}}{\emph{\DUrole{n}{postcodes}}}{}
\pysigstopsignatures
\sphinxAtStartPar
Plot heatmap of max river level and total rainfall in a day
from a given sequence of postcodes.
\begin{quote}\begin{description}
\sphinxlineitem{Parameters}
\sphinxAtStartPar
\sphinxstyleliteralstrong{\sphinxupquote{postcodes}} (\sphinxstyleliteralemphasis{\sphinxupquote{sequence of string}}) \textendash{} Sequence of postcodes.

\sphinxlineitem{Return type}
\sphinxAtStartPar
Folium heatmap object

\end{description}\end{quote}
\subsubsection*{Examples}

\begin{sphinxVerbatim}[commandchars=\\\{\}]
\PYG{g+gp}{\PYGZgt{}\PYGZgt{}\PYGZgt{} }\PYG{n}{mapping}\PYG{o}{.}\PYG{n}{plot\PYGZus{}heatmap}\PYG{p}{(}\PYG{p}{[}\PYG{l+s+s1}{\PYGZsq{}}\PYG{l+s+s1}{DE2 3DA}\PYG{l+s+s1}{\PYGZsq{}}\PYG{p}{,} \PYG{l+s+s1}{\PYGZsq{}}\PYG{l+s+s1}{LN5 7RW}\PYG{l+s+s1}{\PYGZsq{}}\PYG{p}{]}\PYG{p}{)}
\end{sphinxVerbatim}

\end{fulllineitems}

\index{plot\_house\_price() (in module flood\_tool)@\spxentry{plot\_house\_price()}\spxextra{in module flood\_tool}}

\begin{fulllineitems}
\phantomsection\label{\detokenize{index:flood_tool.plot_house_price}}
\pysigstartsignatures
\pysiglinewithargsret{\sphinxcode{\sphinxupquote{flood\_tool.}}\sphinxbfcode{\sphinxupquote{plot\_house\_price}}}{\emph{\DUrole{n}{postcodes}}}{}
\pysigstopsignatures
\sphinxAtStartPar
Plot house price map from given list of postcodes.
\begin{quote}\begin{description}
\sphinxlineitem{Parameters}
\sphinxAtStartPar
\sphinxstyleliteralstrong{\sphinxupquote{postcode}} (\sphinxstyleliteralemphasis{\sphinxupquote{sequence of strs}}) \textendash{} Sequence of postcodes.

\sphinxlineitem{Returns}
\sphinxAtStartPar
\begin{itemize}
\item {} 
\sphinxAtStartPar
\sphinxstyleemphasis{Folium map object} \textendash{} Different colours of circle markers show

\item {} 
\sphinxAtStartPar
\sphinxstyleemphasis{different house price ranges.}

\end{itemize}


\end{description}\end{quote}
\subsubsection*{Examples}

\begin{sphinxVerbatim}[commandchars=\\\{\}]
\PYG{g+gp}{\PYGZgt{}\PYGZgt{}\PYGZgt{} }\PYG{n}{mapping}\PYG{o}{.}\PYG{n}{plot\PYGZus{}house\PYGZus{}price}\PYG{p}{(}\PYG{p}{[}\PYG{l+s+s1}{\PYGZsq{}}\PYG{l+s+s1}{HP27 0BF}\PYG{l+s+s1}{\PYGZsq{}}\PYG{p}{,} \PYG{l+s+s1}{\PYGZsq{}}\PYG{l+s+s1}{W5 2BX}\PYG{l+s+s1}{\PYGZsq{}}\PYG{p}{]}\PYG{p}{)}
\end{sphinxVerbatim}

\end{fulllineitems}

\index{plot\_house\_price\_sampled() (in module flood\_tool)@\spxentry{plot\_house\_price\_sampled()}\spxextra{in module flood\_tool}}

\begin{fulllineitems}
\phantomsection\label{\detokenize{index:flood_tool.plot_house_price_sampled}}
\pysigstartsignatures
\pysiglinewithargsret{\sphinxcode{\sphinxupquote{flood\_tool.}}\sphinxbfcode{\sphinxupquote{plot\_house\_price\_sampled}}}{}{}
\pysigstopsignatures
\sphinxAtStartPar
Plot house price map from given sampled csv
\begin{quote}\begin{description}
\sphinxlineitem{Return type}
\sphinxAtStartPar
Folium map object

\end{description}\end{quote}
\subsubsection*{Examples}

\begin{sphinxVerbatim}[commandchars=\\\{\}]
\PYG{g+gp}{\PYGZgt{}\PYGZgt{}\PYGZgt{} }\PYG{n+nb}{map} \PYG{o}{=} \PYG{n}{plot\PYGZus{}house\PYGZus{}price\PYGZus{}sampled}\PYG{p}{(}\PYG{p}{)}
\end{sphinxVerbatim}

\end{fulllineitems}

\index{plot\_popping() (in module flood\_tool)@\spxentry{plot\_popping()}\spxextra{in module flood\_tool}}

\begin{fulllineitems}
\phantomsection\label{\detokenize{index:flood_tool.plot_popping}}
\pysigstartsignatures
\pysiglinewithargsret{\sphinxcode{\sphinxupquote{flood\_tool.}}\sphinxbfcode{\sphinxupquote{plot\_popping}}}{\emph{\DUrole{n}{postcodes}}}{}
\pysigstopsignatures
\sphinxAtStartPar
Plot popping up window displaying all information required.
\begin{quote}\begin{description}
\sphinxlineitem{Parameters}
\sphinxAtStartPar
\sphinxstyleliteralstrong{\sphinxupquote{postcodes}} (\sphinxstyleliteralemphasis{\sphinxupquote{sequence of string}}) \textendash{} Sequence of postcodes.

\sphinxlineitem{Returns}
\sphinxAtStartPar
\begin{itemize}
\item {} 
\sphinxAtStartPar
\sphinxstyleemphasis{Folium map object} \textendash{} Click on markers will show detailed information on those

\item {} 
\sphinxAtStartPar
\sphinxstyleemphasis{points, including rainfall, rainfall class, river Level,}

\item {} 
\sphinxAtStartPar
\sphinxstyleemphasis{flood event probability, property value and flood risk}

\end{itemize}


\end{description}\end{quote}
\subsubsection*{Examples}

\begin{sphinxVerbatim}[commandchars=\\\{\}]
\PYG{g+gp}{\PYGZgt{}\PYGZgt{}\PYGZgt{} }\PYG{n}{mapping}\PYG{o}{.}\PYG{n}{plot\PYGZus{}popping}\PYG{p}{(}\PYG{p}{[}\PYG{l+s+s1}{\PYGZsq{}}\PYG{l+s+s1}{DE2 3DA}\PYG{l+s+s1}{\PYGZsq{}}\PYG{p}{,} \PYG{l+s+s1}{\PYGZsq{}}\PYG{l+s+s1}{LN5 7RW}\PYG{l+s+s1}{\PYGZsq{}}\PYG{p}{]}\PYG{p}{)}
\end{sphinxVerbatim}

\end{fulllineitems}

\subsubsection*{References}


\renewcommand{\indexname}{Python Module Index}
\begin{sphinxtheindex}
\let\bigletter\sphinxstyleindexlettergroup
\bigletter{f}
\item\relax\sphinxstyleindexentry{flood\_tool}\sphinxstyleindexpageref{index:\detokenize{module-flood_tool}}
\end{sphinxtheindex}

\renewcommand{\indexname}{Index}
\printindex
\end{document}